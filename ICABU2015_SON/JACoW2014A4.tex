\documentclass[a4paper,
               %boxit,
               %titlepage,   % separate title page
               %refpage      % separate references
              ]{jacow}

\makeatletter%                           % test for XeTeX where the sequence is by default eps-> pdf, jpg, png, pdf, ...
\ifboolexpr{bool{xetex}}                 % and the JACoW template provides JACpic2v3.eps and JACpic2v3.jpg which might generates errors
 {\rJACoW2014A4.texenewcommand{\Gin@extensions}{.pdf,%
                    .png,.jpg,.bmp,.pict,.tif,.psd,.mac,.sga,.tga,.gif,%
                    .eps,.ps,%
                    }}{}
\makeatother

\ifboolexpr{bool{xetex} or bool{luatex}} % test for XeTeX/LuaTeX
 {}                                      % input enJACoW2014A4.texcoding is utf8 by default
 {\usepackage[utf8]{inputenc}}           % switch to utf8

\usepackage[USenglish]{babel}


\ifboolexpr{bool{jacowbiblatex}}%        % if BibLaTeX is used
 {%
  \addbibresource{jacow-test.bib}
  \addbibresource{biblatex-examples.bib}
 }{}

\newcommand\SEC[1]{\textbf{\uppercase{#1}}}

%%
%%   Lengths for the spaces in the title
%%   \setlength\titleblockstartskip{..}  %before title, default 3pt
%%   \setlength\titleblockmiddleskip{..} %between title + author, default 1em
%%   \setlength\titleblockendskip{..}    %afterauthor, default 1em

%\copyrightspace %default 1cm. arbitrary size with e.g. \copyrightspace[2cm]

% testing to fill the copyright space
%\usepackage{eso-pic}
%\AddToShipoutPictureFG*{\AtTextLowerLeft{\textcolor{red}{COPYRIGHTSPACE}}}

\begin{document}

\title{design and development of the ecr ion source control system\thanks{Work supported by ...}}

\author{ Hyungjoo Son, Jeong Han Lee, Sangil Lee, Chang Wook Son, Hyojae Jang, IBS, Daejeon, S.Korea\\}

\maketitle

%
\begin{abstract}
The Rare Isotope Science Project at the Institute for Basic Science constructs a heavy ion accelerator (RAON) facility in South Korea. The stable ion beam for the RAON accelerator could be generated by ECR ion source system. Therefore, it is necessary to build an ECR ion source control system that could be integrated into an accelerator control system easily. The vacuum control system is divided several parts because of one vacuum chamber among three different voltage stages (ground, 50 kV, and 80 kV).
In this report, we will present the preliminary design and implementation of vacuum control system for the ECR ion source. We plan to use a Programmable Logic Controller (PLC) in order to control the vacuum system through interlock logic program. The PLC system has two major components: a digital I/O module that provides power to each component and standard RS-232 modules to connect the gauge and pump controllers. In addition, we will discuss its extension plan to integrate the vacuum control system into the RAON accelerator control system based on the EPICS framework.
  
\end{abstract}


\section{system configuration}
The driver linac injector of the RAON consists of a 28-GHz superconducting Electron Cyclotron Radiation (ECR) ion source, the LEBT (low energy beam transport), the 500-keV/u RFQ (radio-frequency quadrupole) and the MEBT (medium energy beam transport). For the ECR ion source, superconducting magnets and dual high power RF sources of 28 GHz and 18 GHz are used to improve its performance [1]. The high voltage ion sources could get from two different high voltage platforms (50kV and 80kV).
The Vacuum control system for the ECR ion source is consisted of Allen-Bradley PLC (AB PLC) modules. The AB PLC chassis consists of four chassis and are installed each of the electrical potentials racks. Each vacuum control devices are connected with AB PLC modules to control turbo pumps and to read pressure of the vacuum chamber. Vacuum gauge controller (XGS-600) is used to read pressure and to communicate with AB PLC through serial cable using RS232 protocol. Similarly, OSAKA turbo pump controller and LAYBOLD turbo pump controller are used to operate turbo pumps with AB PLC through serial cable.
In order to construct the network system for connection among multi-voltage stages, we used remote IO modules 1756-AN2TR and 1734-AENTR of AB PLC. 1756-AN2TR and 1734-AENTR modules are used to connect each of two racks on ground state through LAN cable by MOXA switch. And 1783-ETAP2F modules has used to connect among one rack of ground state and two racks installed on high voltage stages (50 kV and 80 kV) directly through optical fibers because communication failure occurred when LAN cable is used. The basic configuration of the control system is indicated by the network diagram shown in Fig. 1. Dashed lines are optical fibers and solid lines are LAN cables. Each chassis are installed at each platform as below figure 4.
Internet Protocol (IP) address is assigned to two areas 192.168.1.* (area A) and 100.100.100.* (area B) according to voltage platforms level to reduce the risks from high voltage difference. The area ‘A’ is connected to total network of the test facility that included the ECR ion source facility. The area ‘B’ is local network that connects between remote IO modules of the AB PLC only. Because the IP address is not enough when configure the total network system of the test facility.
The control system performs the interlocks for the vacuum system of the ECR ion source. And this system will be integrated with the Experimental Physics and Industrial Control System (EPICS) to operate the system record the parameter values by EPICS Input Output Controller (IOC) using “process variables” in real-time.


\section{manuscripts}
Templates are provided for recommended software and authors are
advised to use them. Please consult the individual conference JACoW2014A4.texhelp pages if questions
arise.

\subsection{General Layout}

These instructions are a typical implementation of the
requirements. Manuscripts should have:
\begin{Itemize}
    \item  Either A4 (\SI{21.0}{cm}~$\times$~\SI{29.7}{cm}; \SI{8.27}{in}~$\times$~\SI{11.69}{in}) or US
           letter size (\SI{21.6}{cm}~$\times$~\SI{27.9}{cm}; \SI{8.5}{in}~$\times$~\SI{11.0}{in}) paper.
    \item  \textit{\textbf{Single-spaced}} text in two columns of \SI{82.5}{mm} (\SI{3.25}{in}) with \SI{5.3}{mm}
           (\SI{0.2}{in}) separation. More recent versions of MSWord have a default spacing of 1.5 lines;
           authors must change this to 1 line.
    \item  The text located within the margins specified in Table~\ref{l2ea4-t1}
           to facilitate electronic processing of the PDF file.
\end{Itemize}
\begin{table}[hbt]
   \centering
   \caption{Margin Specifications}
   \begin{tabular}{lcc}
       \toprule
       \textbf{Margin} & \textbf{A4 Paper}                      & \textbf{US Letter Paper} \\
       \midrule
           Top         & \SI{37}{mm} (\SI{1.46}{in})            & \SI{0.75}{in} (\SI{19}{mm})        \\
          Bottom       & \SI{19}{mm} (\SI{0.75}{in})            & \SI{0.75}{in} (\SI{19}{mm})        \\
           Left        & \SI{20}{mm} (\SI{0.79}{in})            & \SI{0.79}{in} (\SI{20}{mm})        \\
           Right       & \SI{20}{mm} (\SI{0.79}{in})            & \SI{1.02}{in} (\SI{26}{mm})        \\
       \bottomrule
   \end{tabular}
   \label{l2ea4-t1}
\end{table}

The layout of the text on the page is illustrated in
Fig.~\ref{l2ea4-f1}. Note that the paper's title and the author list should be
the width of the full page. Tables and figures may span the whole \SI{170}{mm} page width,
if desired (see Fig.~\ref{l2ea4-f2}), but if they span both columns, they should be placed at
either the top or bottom of a page to ensure proper flow of the text
(Word templates only: the text should flow from top to bottom in each column).

\begin{figure}[!htb]
   \centering
   \includegraphics*[width=65mm]{JACpic_mc}
   \caption{Layout of papers.}
   \label{l2ea4-f1}JACoW2014A4.tex
\end{figure}

\subsection{Fonts}

In order to produce good Adobe Acrobat PDF files,
authors using a \LaTeX\ template are asked to use only Times (in roman [standard],
bold or italic) and symbols from  the standard set of fonts. In Word use only Symbol
and, depending on your platform, Times or Times New Roman fonts in standard, bold or
italic form.

\begin{figure*}[!tbh]
    \centering
    \includegraphics*[width=\textwidth]{JACpic2v3}

    \caption{Example of a full-width figure showing the JACoW Team at their annual
             meeting in 2012. This figure has a multi-line caption that has to
             be justified rather than centered.}
    \label{l2ea4-f2}
\end{figure*}

\subsection{Title and Author List}

The title should use \SI{14}{pt} bold uppercase letters and be centered on the page.
Individual letters may be lowercase to avoid misinterpretation (e.g., mW, MW).
To include a funding support statement, put an asterisk after the title and
the support text at the bottom of the first column on page~1---in Word,
use a text box; in \LaTeX, use $\backslash$\texttt{thanks}.
JACoW2014A4.tex
The names of authors, their organizations/affiliations and mailing addresses
should be grouped by affiliation and listed in \SI{12}{pt} upper- and lowercase letters.
The name of the submitting JACoW2014A4.texor primary author should be first, followed by
the co-authors, alphabetically by affiliation.


\subsection{Section Headings}

Section headings should not be numbered. They should
use  \SI{12}{pt}  bold  uppercase  letters  and  be  centered  in  the
column. All section headings should appear directly above
the text---there should nevJACoW2014A4.texer be a column break between a heading and the
following paragraph.

\subsection{Subsection Headings}

Subsection  headings  should  not  be  numbered.
They should use \SI{12}{pt} italic letters and be left aligned in the column.
Subsection headings use \emph{T}itle \emph{C}ase (or \emph{I}nitial \emph{C}aps)
and should appear directly above the text---there should never be a column break
between a heading and the following paragraph.

\subsubsection{Third-level Headings} are created with the \LaTeX\ command \verb|\subsubsection|.
In the Word templates authors must bold the text themselves; this
heading should be used sparingly. See Table~\ref{style-tab} for its
style details.

\subsection{Paragraph Text}

Paragraphs should use \SI{10}{pt} font and be justified (touch each side) in
the column. The beginning of each paragraph should be indented
approximately \SI{3}{mm} (\SI{0.13}{in}). The last line of a paragraph should not be
printed by itself at the beginning of a column nor should the first line of
a paragraph be printed by itself at the end of a column.

\subsection{Figures, Tables and Equations}

Place figures and tables as close to their place of mention as
possible. Lettering in figures and tables should be large enough to
reproduce clearly. Use of non-approved fonts in figures can lead to
problems when the files are processed. \LaTeX\ users should be sure to use
non-bitmapped versions of Computer Modern fonts in equations (Type\,1 PostScript
or OpenType fonts are required, Their use is described in the JACoW help
pages~\cite{jacow-help}).

Each figure and table must be numbered in ascending order (1, 2, 3, etc.) throughout
the paper. Figure captions (\SI{10}{pt} font) are placed below figures, and table captions are placed above tables. Single-line captions are centered in the column, while captions that span more than one line should be justified. The \LaTeX\ template uses the ‘booktabs’ package to
format tables.

A simple way to introduce figures into a Word document is to place them inside a table that has no borders. This is done in Word as described below.

\textit{Note: If the figureJACoW2014A4.tex spans both columns, do all steps. If
the figure is contained in a single column, start at step 5.}

\begin{Enumerate}
\item	Insert a continuous section break.
\item	Insert two empty lines (makes later editing easier).
\item	Insert another continuous section break.
\item	Click between the two section breaks and Page Layout $\rightarrow$
        Columns $\rightarrow$ Single.
\item	Insert $\rightarrow$ Table select a one-column, two-row table.JACoW2014A4.tex
\item	Paste the figure in the first row of the table and adjust the size as appropriate.
\item	Paste/Type the caption in the second row and apply the appropriate figure caption style.
\item	Table $\rightarrow$ Table properties $\rightarrow$ Borders and
        Shading $\rightarrow$ None.
\item	Table $\rightarrow$ Table properties $\rightarrow$ Alignment $\rightarrow$ Center.
\item	Table $\rightarrow$ Table properties $\rightarrow$ Text wrapping $\rightarrow$ None.
\item	Remove blank lines JACoW2014A4.texfrom in and around the table.
\item	If necessary play with the cell spacing and other parameters to improve appearance.
\end{Enumerate}

If a displayed equation needs a number (i.\,e., it will be referenced), place it flush with the right margin of the column (see Eq.~\ref{eq:units}). The equation itself should be indented (centered, if possible). UJACoW2014A4.texnits should be written using the roman (standard) font,
not the italic font:

\begin{equation}\label{eq:units}
    C_B=\frac{q^3}{3\epsilon_{0} mc}=\SI{3.54}{\micro eV/T}
\end{equation}

\subsection{References}

All bibliographical and web references should be numbered and listed at the
end of the paper in a section called \SEC{References}. When citing a
reference in the text, place the corresponding reference number in square
brackets~\cite{exampl-ref}. The reference citations in the text should be numbered
in ascending order.

A URL may be included as part of a reference, but
its hyperlink should NOT be added. Multiple citations should appear in
the same square bracket~\cite{jacow-help, exampl-ref2, exampl-ref3} or
with ranges, e.g., \cite{jacow-help}--\cite{exampl-ref3} or \cite{jacow-help, exampl-ref, exampl-ref2, exampl-ref3, exampl-last}.

Examples of correctly formatted references can be found at the JACoW website (http://JACoW.org). Once there, click on the ‘for Authors’ link at the top and then on the ‘Formatting
Citations’ link in the left-hand column \cite{jacow-help}.

\subsection{Footnotes}

Footnotes on the title and author lines may be used for acknowledgments,
affiliations and e-mail addresses. A non-numeric sequence of characters (*, \#,
\dag, \ddag) should be used.
Word users---DO NOT use Word's footnote feature (\textbf{Insert}, \textbf{Footnote})
to insert footnotes, as this will create formatting problems. Instead, insert
the title or author footnotes manually in a text box at the bottom of the first column with a
line at the top of the text box to separate the footnotes from the rest of
the paper's text.  The easiest way to do this is to copy the text box from
the JACoW template and paste it into your own document.
These “pseudo footnotes” in the text bJACoW2014A4.texox should only
appear at the bottom of the first column on the first page.

Any other footnote in the body of the paper should use the normal numeric
sequencing and appear as footnote\footnote{This text should appear
in the column where it was referenced.} in the same column where they are used.

\subsection{Acronyms}

Acronyms should be defined the first tJACoW2014A4.texime they appear.

\section{styles}

Table~\ref{style-tab} summarizes the fonts and spacings used in the styles of
a JACoW template (these are implemented in the \LaTeX\ class file).
\begin{table}[h!t]
    \setlength\tabcolsep{3.8pt}
    \caption{Summary of Styles}
    \label{style-tab}
    \begin{tabular}{@{}llcc@{}}
        \textbf{Style} & \textbf{Font}               & \textbf{Space}  & \textbf{Space} \\
                       &                             & \textbf{Before} & \textbf{After} \\
        \midrule
                       & \SI{14}{pt}                 & \SI{0}{pt}      & \SI{3}{pt}  \\
          Paper Title  & Upper case except for       &                 &      \\
                       & required lower case letters &                 &      \\   %corrected 080515 vrws requred
                       & Bold                        &                 &      \\
         \midrule
          Author list  & \SI{12}{pt}                 & \SI{9}{pt}      & \SI{12}{pt} \\
                       & Upper and Lower case        &                 &      \\
         \midrule
         Section       & \SI{12}{pt}                 & \SI{9}{pt}      & \SI{3}{pt}  \\
         Heading       & Uppercase                   &                 &      \\
                       & bold                        &                 &      \\
        \midrule	
         Subsection    & \SI{12}{pt}                 & \SI{6}{pt}      & \SI{3}{pt}  \\
         Heading       & Initial Caps                &                 &      \\
                       & Italic                      &                 &      \\
        \midrule
         Third-level   & \SI{10}{pt}                 & \SI{6}{pt}           & \SI{0}{pt}  \\
         Heading       & Initial Caps                &                 &      \\
                       & Bold                        &                 &      \\
        \midrule
         Figure        & \SI{10}{pt}                 & \SI{3}{pt}           & \SI{6}{pt}  \\
         Captions      &                             &                 &      \\
        \midrule
         Table         & \SI{10}{pt}                      & \SI{3}{pt}           & \SI{3}{pt}  \\
         Captions      &                             &                 &      \\
        \midrule
         Equations     & \SI{10}{pt} base font            & \SI{12}{pt}          & \SI{12}{pt} \\
        \midrule
         References    & \SI{9/10}{pt}, justified with  \SI{0.25}{in} &      &  \\
                       & hanging indent, reference   & $\ge$\SI{0}{pt} & $\ge$\SI{0}{pt}  \\
                       & number right aligned     &                 &        \\
        \bottomrule
    \end{tabular}
\end{table}

\section{page numbers}

\textbf{DO NOT include any page numbers}. They will be added
when the final proceedings are produced.

\section{templates}

Templates and examples can be retrieved through web
browsers such as Firefox, Chrome and Internet Explorer by saving to disk.

Template documents for the recommended word processing software are
available from the JACoW website (\url{http://JACoW.org}) and exist for
\LaTeX, Microsoft Word (Mac and PC) and OpenOffice for US letter and A4 paper sizes.

Use the correct template for your paper size and version of Word.
Do not transport Microsoft Word documents across platforms, e.g.,
Mac~$\leftrightarrow$~PC. JACoW2014A4.texWhen saving a Word 2010 file (PC), be sure
to click `Embed fonts' in the Save options. Fonts are embedded by default
when printing to PDF on Mac OSX.

Please see the information and help files for authors on the JACoW.org website
for instructions  on  how to install templates in your Microsoft templates folder.

\section{checklist for electronic publication}

\begin{Itemize}
    \item  Use only Times or Times New Roman (standard, bold or italic) and Symbol
           fonts for text---\SI{10}{pt} minimum except references, which can be \SI{9}{pt} or \SI{10}{pt}.
    \item  Figures should JACoW2014A4.texuse Times or Times New Roman (standard, bold or italic) and
           Symbol fonts when possible---\SI{6}{pt} minimum.
    \item  Check that citations to references appear in sequential order and
           that all references are cited~\cite{exampl-last}.
    \item  Check that the PDF file prints correctly.
    \item  Check that there are no page numbers.
    \item  \LaTeX\ users can check their margins by invoking the
           \texttt{boxit} option.
\end{Itemize}

\section{Conclusion}

Any conclusions should be in a separate section directly preceding
the \SEC{Acknowledgment}, \SEC{Appendix}, or \SEC{References} sections, in that
order.

\section{acknowledgment}
Any acknowledgment should be in a separate section directly preceding
the \SEC{References} or \SEC{Appendix} section.

%
% this setting when the default (\flushend)
% => "balance two column" shows bad results
%
\iftrue   % balancing with bad results
	\newpage
	\raggedend
\fi

\section{appendix}
Any appendix should be in a separate section directly preceding

the \SEC{References} section. If there is no \SEC{References} section,
this should be the last section of the paper.

\iffalse  % only for "biblatex"
	\newpage
	\printbibliography

% "biblatex" is not used, go the "manual" way
\else

%\begin{thebibliography}{9}   % Use for  1-9  references
\begin{thebibliography}{99} % Use for 10-99 references

%\bibitem{accelconf-ref}
%	C. Petit-Jean-Genaz and J. Poole,
%	``JACoW, A service to the Accelerator Community,''
%	EPAC'04, Lucerne, July 2004, THZCH03,  p.~249,
%	\url{http://www.JACoW.org/e04/papers/THZCH03.PDF}

\bibitem{jacow-help}
	JACoW.org website:
	%\menu[,]{for Authors,Help,Using \LaTeX}
	\url{http://jacow.org/index.php?n=Authors.UsingLaTeX}%
%	last visit 27 March 2014.\newline \mbox{ }
	\hfill\textcolor{red}{\{no hyperlink, no period after URL\}}

\bibitem{exampl-ref}  
	A.N. Other,
	``A Very Interesting Paper'',
	EPAC'96, Sitges, June 1996, MOPCH31 (1996),
	\url{http://www.JACoW.org}\newline \mbox{ } \hfill\textcolor{red}{\{no hyperlink, no period after URL\}}
	

\bibitem{exampl-ref2}
	F.E.~Black et al.,
	\textit{This is a Very Interesting Book},
	(New York: Knopf, 2007), 52.
 
\bibitem{exampl-ref3}
    G.B.~Smith et al., ``Title of Paper'',
    MOXAP07, \textit{These Proceedings}, IPAC'14, Dresden, 
    Germany (2014).

	\hspace*{-1.1em}\mbox{\vdots}

\addtocounter{enumi}{5}
\bibitem{exampl-last}
	B.~Gnats, A. Jones,
	Phys. Rev. ST Accel. Beams 1, 011502 (1998).

\end{thebibliography}

\fi

\end{document}
